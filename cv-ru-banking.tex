%!TEX output_directory=dist
\documentclass[11pt,a4paper,russian]{moderncv}
\usepackage[T2A]{fontenc}
\usepackage[utf8]{inputenc}               % Кодировка utf8
\usepackage[english, russian]{babel}
\usepackage[scale=0.75]{geometry}
\usepackage{lmodern}
\moderncvstyle{banking}
\moderncvcolor{blue}

%\usepackage[T1]{fontenc}

\usepackage{graphicx} %maybe obsolete, but no time to check it out
\usepackage[margin=10pt,font=small,labelfont=bf]{caption} %for figures
\setlength{\hintscolumnwidth}{4cm}
%personal data
\firstname{Дмитрий}
\familyname{Шихалеев}
%\title{Curriculum vitae}
\address{Санкт-Петербург}
\mobile{+79602408223}
\email{dmitriy.shihaleev@gmail.com}
\social[github]{Padavan}
%\photo[64pt]{photoCV.jpg}
%\nopagenumbers{}                             % uncomment to suppress automatic page numbering for CVs longer than one page

\newcommand{\cvdoublecolumn}[2]{%
  \cvitem[0.75em]{}{%
    \begin{minipage}[t]{\listdoubleitemmaincolumnwidth}#1\end{minipage}%
    \hfill%
    \begin{minipage}[t]{\listdoubleitemmaincolumnwidth}#2\end{minipage}%
    }%
}

% usage: \cvreference{name}{address line 1}{address line 2}{address line 3}{address line 4}{e-mail address}{phone number}
% Everything but the name is optional
% If \addresssymbol, \emailsymbol or \phonesymbol are specified, they will be used.
% (Per default, \addresssymbol isn't specified, the other two are specified.)
% If you don't like the symbols, remove them from the following code, including the tilde ~ (space).

\newcommand{\cvreference}[7]{%
    \textbf{#1}\newline% Name
    \ifthenelse{\equal{#2}{}}{}{\addresssymbol~#2\newline}%
    \ifthenelse{\equal{#3}{}}{}{#3\newline}%
    \ifthenelse{\equal{#4}{}}{}{#4\newline}%
    \ifthenelse{\equal{#5}{}}{}{#5\newline}%
	 \ifthenelse{\equal{#6}{}}{}{\texttt{#6}\newline}%
    %\ifthenelse{\equal{#6}{}}{}{\emailsymbol~\texttt{#6}\newline}%
    \ifthenelse{\equal{#7}{}}{}{\phonesymbol~#7}}



\begin{document}
\maketitle

\section{Цель}

\cvitem{Получить должность в области информационных технологий в направлениях, связанных с анализом данных и обработкой сигналов. Получить опыт для дальнейшего роста и развития.}{	 }{}

\section{Образование}

\cventry{2008 - 2016}{Степень бакалавра по направлению "Электроника и наноэлектроника"}{Санкт-Петербургский политехнический университет Петра Великого}{Санкт-Петерубрг}{}{}{}

\section{Опыт работы}

\cventry{2014--2016}{Специалист поддержки}{Национальные кабельные сети}{Санкт-Петербург}{}{}


\section{Компьютерные навыки}
\cvitemwithcomment{\LaTeX}{Несколько лет использования}{Написание письменных работ в университете }
\cvitemwithcomment{R}{Регулярное использования}{Статистический анализ экспериментальных данных, импортирование и обработка. Использование библиотеки caret для машинного обучения и библиотеки knitr для генерации докладов  }
\cvitemwithcomment{Linux}{Отличные знания}{Несколько лет использования}
\cvitemwithcomment{C}{Базовые знания}{Небольшие эксперименты с ARM микроконтроллерами}
\cvitemwithcomment{Python}{Базовые знания}{Статистический анализ данных, разработка ПО с использованием тулкитов Tk/Qt4/Qt5}
\cvitemwithcomment{JavaScript}{Базовые знания}{Простые Node.js приложения}
\cvitemwithcomment{Другие средства и инструменты}{}{GNU Toolchain(gcc,gdb, valgrind), системы контроля версий (git), средства автоматической сборки (cmake make), GNU coreutils, базовые знания bash скриптинга, хорошая математическая подготовка, владею средствами пайки и монтажа.}



\section{Языки}

\cvitemwithcomment{Русский}{Родной}{}
\cvitemwithcomment{Английский}{Технический - свободно, разговорный - средний уровень}{}




\end{document}\textsl{}
